\section{Related Work}
\label{sec:related}


Optimizing meshes for fit point cloud data has been approached through vertex additions and removals~\cite{hoppe:1994}, although this work assumes the point data are precise and dense.

In our work,building meshes on the segmented leaf boundaries of the rgb imagery of the sensor serve as an important precursor to fitting mesh surfaces on 3D point clouds. Since the rgb imagery is itself captured at maximum resolution (720 x 1280) of the sensor,it is expected to preserve several important saliency cues like homogeneity of color, texture and edges in local regions. Our segmentation approach aims to capture several variations and deformations of the leaf shape that can lead to more precise mesh fitting of leaves on 3D point clouds. We aim to obtain a fairly unconstraint leaf fitting on the 3D point clouds without using a-priori shape information or explicit parametrization or categorization of leaf. In this context, it should be noted that our main goal is not leaf segmentation but to obtain precise mesh fitting on 3D point clouds by using the segmented leaf and plant boundaries on rgb image. Complex leaf structures like overlapping leaves and cluttered background are not dealt with at this stage, and the main aim of segmentation is to extract the leaf regions from the non-leaf regions appearing in the background. The challenges of segmentation in our case is color variability due to lighting changes, shadows, specular reflection and natural variations in leaf color. Although there are plenty of contour detection and segmentation in the literature that deals with boundaries and contour detection of objects, 2D and 3D leaf shape and boundary segmentation and detections are relatively new. Simpler cues like color, texture and local brightness and its combination are extensively used in contour and image boundary detections ~\cite{martin2004learning,valliammal2012leaf}. To enhance segmentation and boundary regions, active contours and several improved variants of it ~\cite{mishra2011decoupled} are also found to be quite popular in computer vision community. Manh et al. ~\cite{Manh2001139} use deformable templates in weed leaf segmentation which consists of fitting a parametric models to leaf outlines in image, by minimizing energy term related to internal constraints of the model and salient features of the image, such as color of plant. It fits to the more recently published work of Toshev et al. ~\cite{toshev2012shape} which uses both geometric properties of object boundary edges and coherent saliency cues distinct from the background. The method tries to overcome clutter in realistic scenes by handling segmentation using object specific knowledge like similarity in shape (top-down approach) and region growing principles (bottom up approach) from cues like color texture and normalized perimeter of objects. Both of them require extensive computational requirements and time. LeafSnap~\cite{kumar2012leafsnap}, a popular mobile application tool that identifies leaf species from photographs of leafs,rely on the more simplistic method of segmenting leaf by estimating the foreground and background color distribution by using Expectation Maximization in the saturation value space of the HSV colorspace. Research has also gained particular attention on segmentation of overlapping leaves for leaf detection. Pape et al. ~\cite{Pape2015} tackles the problem of segmenting overlapping leaves partly by thresholding on built 3D histograms cubes of Lab color space both for training and testing data, and the quality of segmentation depends on the homogeneity of training data. But for large overlaps, it fails to detect leaf structures. Wei Ma et. al ~\cite{ma2008image} proposes modelling leaf from a different domain; voxels. It aims to deal with the highly cluttered background by extracting the more visible apex features (position, middle axis and neighborhood) of the leaves instead of intelligent segmentation, from the volumetric data recovered from the images. According to the paper, volumetric data provides pose and position of 3D generic leaves, and 3d leaf shapes can be extracted based on the optimized voxels.


