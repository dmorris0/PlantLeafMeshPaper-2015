\section{Related Work}
\label{sec:related}


Optimizing meshes for fit point cloud data has been approached through vertex additions and removals~\cite{hoppe:1994}, although this work assumes the point data are precise and dense.

Although there are plenty of contour detection and segmentation in the literature that deals with boundaries and contour detection of objects, 2D and 3D leaf shape and boundary segmentation and detections are relatively new. Simpler cues like color, texture and local brightness and its combination are extensively used in contour and image boundary detections ~\cite{martin2004learning,valliammal2012leaf}. To enhance segmentation and boundary regions, active contours and several improved variants of it ~\cite{mishra2011decoupled} are also found to be quite popular in computer vision community. Manh et al. ~\cite{Manh2001139} use deformable templates in weed leaf segmentation which consists of fitting a parametric models to leaf outlines in image, by minimizing energy term related to internal constraints of the model and salient features of the image, such as color of plant. It fits to the more recently published work of Toshev et al. ~\cite{toshev2012shape} which uses both geometric properties of object boundary edges and coherent saliency cues distinct from the background. The method obtains claims to overcome clutter in realistic scenes by handling segmentation using object specific knowledge like similarity in shape (top-down approach) and region growing principles (bottom up approach) from cues like color texture and normalized perimeter. In 3D detection, Wei Ma et. al ~\cite{ma2008image} proposed extracting the apex points of the leaves from the volumetric data recovered from the images. According to the paper, volumetric data provides pose and position of 3D generic leaves, and 3d leaf shapes can be extracted based on the optimized voxels.


