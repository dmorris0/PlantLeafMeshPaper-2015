\section{Related Work}
\label{sec:related}



%Optimizing meshes for fit point cloud data has been approached through vertex additions and removals~\cite{Hoppe1994}, although this work assumes the point data are precise and dense.

This work detects and segments plant leaves in color images before fitting a $3$D mesh.  These detection and segmentation tasks can be challenging depending on the clutter, lighting, shadows, overlapping leaves, and variations in leaf.  There have been numerous research efforts in this area to address these challenges.

One class of approaches to leaf segmentation is through combining many low-level features.  Contour detection is well studied although only more recently have they been used for leaf shape and boundary segmentation.  Simple cues like color, texture and local brightness and its combination are extensively used in contour and image boundary detections~\cite{martin2004learning,valliammal2012leaf}.  Active contours can enhance segmentation and boundary regions~\cite{mishra2011decoupled}. LeafSnap~\cite{kumar2012leafsnap}, a popular mobile application tool that identifies leaf species from photographs of leafs, relies on  segmenting leaf by estimating the foreground and background color distribution using Expectation Maximization in the saturation value space of the HSV colorspace.  Pape et al.~\cite{Pape2015} tackles the problem of segmenting overlapping leaves partly by thresholding on built 3D histograms cubes of Lab color space.

Another class of segmentation methods uses prior models.  Manh et al.~\cite{Manh2001139} use deformable templates in weed leaf segmentation which consists of fitting a parametric models to leaf outlines in image, by minimizing energy term related to internal constraints of the model and salient features of the image, such as color of plant. In related work, Toshev et al.~\cite{toshev2012shape} use both geometric properties of object boundary edges and coherent saliency cues distinct from the background. The method overcomes clutter in realistic scenes by handling segmentation using object specific knowledge like similarity in shape (top-down approach) and region growing principles (bottom up approach).  

A third approach includes Wei Ma et. al ~\cite{ma2008image} who proposes modeling leaves using voxels. It aims to deal with the highly cluttered background by extracting the more visible apex features (position, middle axis and neighborhood) of the leaves instead of intelligent segmentation, from the volumetric data recovered from the images.  Volumetric data provide pose and position of $3$D leaves, and the $3$D leaf shapes can be extracted from the optimized voxels.

The focus of this work is accurate leaf boundary and surface estimation, rather than robust segmentation.  We restrict ourselves to relatively simple conditions including non-overlapping leaves, low background clutter, and single plant types.    On the other hand we do not want to constrain leaf boundaries with prior parametric shape models which could prevent accurate leaf shape estimation for leaves that do not fit the models. Instead we rely primarily on color cues~\cite{achanta2012slic} for segmentation.

Mesh fitting to range data has been well studied.  Much work address how to combine range maps from different perspectives including early work on zippered polygons~\cite{Turk1994}.  The method developed by Curless and Levoy~\cite{Curless:1996}, and built on by later work~\cite{Izadi:2011,Newcombe:2011}, populates a weighted voxel occupancy grid from the depth data and recovers the surface by triangulating an isosurface.  Advantages of this method include that surface topology is automatically determined, additional data can be readily incorporated, and it incorporates directional uncertainty of range data into the models.  Recent approaches for environment modeling from RGB-D cameras including ~\cite{Xu2011} and Kinect Fusion~\cite{Izadi:2011,Newcombe:2011} build on this voxel modeling and accumulation.  However in our application the sensor is fixed and there is no option for merging views from different perspectives.  In addition these methods work room-size scales and the sensor noise is significantly smaller relative to object size than in our application.  

Foliage estimation for graphics applications such as~\cite{Quan:2006,Bradley:2013} rely heavily on fitting prior models, and the plant inspection method in~\cite{Alenya2011,Alenya2013} fits simple geometric models, such as parabolas, to leafs.  These models are not flexible enough to model curved leaves with complicated shapes.  And it is this task of estimating the surface variation of plant leaves that is the focus of this work.


