\section{Related Work}
\label{sec:related}


This paper builds on past work in the areas of leaf detection, segmentation and $3$D shape fitting.   Detection and segmentation of leaves can be challenging and there have been numerous research efforts in this area.

One class of approaches to leaf segmentation is through combining many low-level features.  Simple cues like color, texture and local brightness and its combination are extensively used in contour and image boundary detections~\cite{martin2004learning,valliammal2012leaf}.  Active contours can enhance segmentation and boundary regions~\cite{mishra2011decoupled}. LeafSnap~\cite{kumar2012leafsnap}, identifies leaf species from photographs of leafs and segments leafs uses Expectation Maximization of color distributions in the HSV colorspace.  Pape et al.~\cite{Pape2015} segment overlapping leaves using 3D histograms cubes in the Lab color space.

Another class of segmentation methods uses parametric models.  Manh et al.~\cite{Manh2001139} use deformable templates in weed leaf segmentation which consists of fitting parametric models to leaf outlines in image, by minimizing energy term related to internal constraints of the model and salient features of the image, such as color of plant. In related work, Toshev et al.~\cite{toshev2012shape} use both geometric properties of object boundary edges and coherent saliency cues distinct from the background.

A third approach includes Wei Ma et. al ~\cite{ma2008image} who proposes modeling leaves using voxels. It aims to deal with the highly cluttered background by extracting the more visible apex features (position, middle axis and neighborhood) of the leaves instead of intelligent segmentation, from the volumetric data recovered from the images.  Volumetric data provide pose and position of $3$D leaves, and the $3$D leaf shapes can be extracted from the optimized voxels.

The focus of this work is accurate leaf boundary and surface estimation, rather than robust segmentation.  We restrict ourselves to relatively simple conditions including non-overlapping leaves, low background clutter, and single plant types.    On the other hand we do not want to constrain leaf boundaries with prior parametric shape models which could prevent accurate leaf shape estimation for leaves that do not fit the models. Hence we rely primarily on color cues~\cite{achanta2012slic} for segmentation.

Mesh fitting to range data has been well studied.  Much work address how to combine range maps from different perspectives notably early work on zippered polygons~\cite{Turk1994}, and voxel accumulation with iso-surface estimation developed by Curless and Levoy~\cite{Curless:1996}.  The latter method populates a weighted voxel occupancy grid from the depth data and recovers the surface by triangulating an iso-surface.  Advantages of this method include that surface topology is automatically determined, additional data can be readily incorporated, and it incorporates directional uncertainty of range data into the models.  Recent approaches for environment modeling from RGB-D cameras including ~\cite{Xu2011} and Kinect Fusion~\cite{Izadi:2011,Newcombe:2011}.  These voxel methods work well on room-size scales where the sensor noise is significantly smaller relative to object size than in our application.

Model-based leaf estimation is another approach to our problem.  Some methods~\cite{Quan:2006,Bradley:2013} fit leaf templates to sparse $3$D data.  Another method is to fit geometric models, such as parabolas, to leafs\cite{Alenya2011,Alenya2013}.  However, none of these models are flexible enough to capture curved leaves with complicated shapes.  And it is this task of estimating the surface shape variation of plant leaves that is the focus of this work.


